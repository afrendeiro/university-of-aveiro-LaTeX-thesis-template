%
% uaThesis example (for a thesis written in Portuguese)
%
% the complete list of options and commands can be found in uaThesis.sty 
%

\documentclass[11pt,twoside,a4paper]{report}
\usepackage[DETI,newLogo]{uaThesis}

\def\ThesisYear{2012}

% optional packages
\usepackage[portuguese]{babel}
\usepackage{hyperref}
\usepackage{amsmath}
\usepackage{amssymb}
\usepackage{xspace}% used by \sigla

% optional (comment to use default)s
%   depth of the table of contents
%     1 ... chapther and sections
%     2 ... chapters, sections, and subsections
%     3 ... chapters, sections, subsections, and subsubsections
\setcounter{tocdepth}{3}

% optional (comment to used default)
%   horizontal line to separate floats (figures and tables) from text
\def\topfigrule{\kern 7.8pt \hrule width\textwidth\kern -8.2pt\relax}
\def\dblfigrule{\kern 7.8pt \hrule width\textwidth\kern -8.2pt\relax}
\def\botfigrule{\kern -7.8pt \hrule width\textwidth\kern 8.2pt\relax}

% custom macros (could also be defined using \newcommand)
\def\I{\mathtt{i}}         % one possible way to represent $\sqrt{-1}$
\def\Exp#1{e^{2\pi\I #1}}  % argument inside braces, i.e., "{}"
\def\EXP#1.{e^{2\pi\I #1}} % argument finishes when a full stop is encountered, i.e., "."
\def\sigla{\LaTeX\xspace}  % use as "blabla \sigla blabla (no need to do "blabla \sigla\ blabla"

\def\AddVMargin#1{\setbox0=\hbox{#1}%
                  \dimen0=\ht0\advance\dimen0 by 2pt\ht0=\dimen0%
                  \dimen0=\dp0\advance\dimen0 by 2pt\dp0=\dimen0%
                  \box0}   % add extra vertical space above and below the argument (#1)
\def\Header#1#2{\setbox1=\hbox{#1}\setbox2=\hbox{#2}%
           \ifdim\wd1>\wd2\dimen0=\wd1\else\dimen0=\wd2\fi%
           \AddVMargin{\parbox{\dimen0}{\centering #1\\#2}}} % put #1 on top #2


\begin{document}

%
% Cover page (use only one of the first two \TitlePage)
%

% First alternative, with a figure
\TitlePage
  %\GRID  % for debugging ONLY
  \HEADER{\BAR\FIG{\includegraphics[height=60mm]{uaLogoOld}}} % the \FIG{} is optional
         {\ThesisYear}
  \TITLE{Z\'e \newline Manel}
        {Como escrever uma tese bonita e cheia de resultados importantes}
\EndTitlePage
\titlepage\ \endtitlepage % empty page

% Second alternative, with a citation
\TitlePage
  %\GRID  % for debugging ONLY
  \HEADER{\BAR\FIG{\begin{minipage}{50mm} % no more than 120mm
          ``I'm King of the world.''
           \begin{flushright}
            --- Jack Nicholson
           \end{flushright}
          \end{minipage}}}
         {\ThesisYear}
  \TITLE{Z\'e \newline Manel}
        {Como escrever uma tese bonita e cheia de resultados importantes}
\EndTitlePage
\titlepage\ \endtitlepage % empty page


%
% Initial thesis pages
%

\TitlePage
  \HEADER{}{\ThesisYear}
  \TITLE{Z\'e \newline Manel}
        {Como escrever uma tese bonita e cheia de resultados importantes}
  \vspace*{15mm}
  \TEXT{}
       {Disserta\c c\~ao apresentada \`a Universidade de Aveiro para cumprimento dos requesitos
        necess\'arios \`a obten\c c\~ao do grau de Doutor em X, realizada sob a orienta\c c\~ao
        cient\'\i fica de Y, Professor do Departamento Z da Universidade de Aveiro}
\EndTitlePage
\titlepage\ \endtitlepage % empty page

\TitlePage
  \vspace*{55mm}
  \TEXT{\textbf{o j\'uri~/~the jury\newline}}
       {}
  \TEXT{presidente~/~president}
       {\textbf{ABC}\newline {\small
        Professor Catedr\'atico da Universidade de Aveiro (por delega\c c\~ao da Reitora da
        Universidade de Aveiro)}}
  \vspace*{5mm}
  \TEXT{vogais~/~examiners committee}
       {\textbf{DEF}\newline {\small
        Professor Catedr\'atico da Universidade de Aveiro (orientador)}}
  \vspace*{5mm}
  \TEXT{}
       {\textbf{GHI}\newline {\small
        Professor associado da Universidade J (co-orientador)}}
  \vspace*{5mm}
  \TEXT{}
       {\textbf{KLM}\newline {\small
        Professor Catedr\'atico da Universidade N}}
\EndTitlePage
\titlepage\ \endtitlepage % empty page

\TitlePage
  \vspace*{55mm}
  \TEXT{\textbf{agradecimentos~/\newline acknowledgements}}
       {\'E com muito gosto que aproveito esta oportunidade para agradecer a todos os que me
        ajudaram durante este longos e penosos anos, cheios de altos e baixos (mais baixos que
        altos)\ldots}
  \TEXT{}
       {Desejo tamb\'em pedir desculpa a todos que tiveram de suportar o meu desinteresse pelas
        tarefas mundanas do dia-a-dia, \ldots}
\EndTitlePage
\titlepage\ \endtitlepage % empty page

\TitlePage
  \vspace*{55mm}
  \TEXT{\textbf{Resumo}}
       {Nos dias que correm, \'e frequente um trabalho ser avaliado pela sua apar\^encia em vez de
        o ser pelo seu conte\'udo. Sendo assim, sem descurar este \'ultimo, nesta tese descrevemos
        maneiras revolucion\'arias de transformar um documento s\'olido e austero num documento
        s\'olido e belo, capaz de fazer chorar de alegria (ou de inveja) qualquer leitor, mesmo
        quando este n\~ao percebe nada do que l\'a est\'a escrito.}
  \TEXT{}
       {A explora\c c\~ao de novas descobertas na \'area da percep\c c\~ao visual, nomeadamente
        no que se refere \`a aprecia\c c\~ao de obras de arte geniais, \ldots}
\EndTitlePage
\titlepage\ \endtitlepage % empty page

\TitlePage
  \vspace*{55mm}
  \TEXT{\textbf{Abstract}}
       {Nowadays, it is usual to evaluate a work \ldots}
\EndTitlePage
\titlepage\ \endtitlepage % empty page


%
% Tables of contents, of figures, ...
%

\pagenumbering{roman}
\tableofcontents

\cleardoublepage
\listoffigures

\cleardoublepage
\listoftables


% The chapters (usually written using the isolatin font encoding ...)

\cleardoublepage
\pagenumbering{arabic}
\chapter{Introdu\c c\~ao}

Para este tipo de documentos, o autor prefere o estilo \verb+report+ ao estilo \verb-book-,
pelo que somente o primeiro \'e suportado oficialmente pelo ficheiro \verb:uaThesis.sty:.
\'E poss\'\i vel for\c car um novo cap\'\i tulo a come\c car numa p\'agina \'\i mpar atrav\'es
do uso do comando \verb!\cleardoublepage!. Deve-se sempre incluir a op\c c\~ao \verb+a4paper+
para especificar as dimens�es das folhas de papel.

Escusado ser\'a dizer (na realidade, escrever) que se a l\'\i ngua em que est\'a escrito o
documento n\~ao for o Ingl\^es, ser\'a preciso utilizar o ``pacote'' \verb.babel..


\section{Op\c c\~oes}

Apresentamos de seguida, uma lista das op\c c\~oes suportadas.
\begin{itemize}
  \item \verb+oldLogo+: usa o ``antigo'' logotipo da Universidade de Aveiro.
  \item \verb+newLogo+: usa o ``novo'' logotipo da Universidade de Aveiro.
  \item \verb+final+: \textbf{n\~ao escreve} o texto ``documento provis\'orio'' na capa: al\'em
        disso, todas as marcas que assinalam linhas demasiado compridas s\~ao eliminadas.
  \item \verb+DETI+, \verb+DM+, \verb+DF+: para teses escritas por alunos dos departamentos de
        electr\'onica, telecomunica\c c\~oes e inform\'atica, de matem\'atica, e de f\'\i sica.
        \'E muito f\'acil incluir uma op\c c\~ao para um outro departamentos editando o
        ficheiro \verb+uaThesis.sty+.
\end{itemize}


\section{Problemas conhecidos}

N\~ao h\'a problemas conhecidos. Todas as coisas aparentemente erradas n\~ao s\~ao problemas
(\textit{bugs}), s\~ao esquisitices (\textit{features}) do ficheiro de estilo \verb:uaThesis.sty:.


\cleardoublepage
\chapter{Alguns truques \'uteis}

\'E poss\'\i vel desalinhar texto na vertical, para cima
  \raise 2pt\hbox{(aqui est\'a ele para cima)}
e para baixo
  \lower 2pt\hbox{(aqui est\'a ele para baixo)}.
Atendendo que o texto desalinhado est\'a dentro de uma caixa (\verb+\hbox+), este
ficar\'a sempre na mesma linha, ou seja, uma linha s\'o poder\'a ser partida no
in\'\i cio ou no fim desse texto. No entanto,
\raise -2.25pt\hbox{isso}
\raise -1.75pt\hbox{pode}
\raise -1.25pt\hbox{ser}
\raise -0.75pt\hbox{resolvido}
\raise -0.25pt\hbox{colocando}
\raise  0.25pt\hbox{o}
\raise  0.75pt\hbox{texto}
\raise  1.25pt\hbox{em}
\raise  1.75pt\hbox{caixas}
\raise  2.25pt\hbox{separadas}.
Os mais malucos podem at\'e experimentar coisas do g\'enero
\raise  0.000pt\hbox{$\cdot$}%
\raise  1.236pt\hbox{$\cdot$}%
\raise  2.351pt\hbox{$\cdot$}%
\raise  3.236pt\hbox{$\cdot$}%
\raise  3.804pt\hbox{$\cdot$}%
\raise  4.000pt\hbox{$\cdot$}%
\raise  3.804pt\hbox{$\cdot$}%
\raise  3.236pt\hbox{$\cdot$}%
\raise  2.351pt\hbox{$\cdot$}%
\raise  1.236pt\hbox{$\cdot$}%
\raise  0.000pt\hbox{$\cdot$}\-%
\raise -1.236pt\hbox{$\cdot$}%
\raise -2.351pt\hbox{$\cdot$}%
\raise -3.236pt\hbox{$\cdot$}%
\raise -3.804pt\hbox{$\cdot$}%
\raise -4.000pt\hbox{$\cdot$}%
\raise -3.804pt\hbox{$\cdot$}%
\raise -3.236pt\hbox{$\cdot$}%
\raise -2.351pt\hbox{$\cdot$}%
\raise -1.236pt\hbox{$\cdot$}%
\raise  0.000pt\hbox{$\cdot$}\-%
\raise  1.236pt\hbox{$\cdot$}%
\raise  2.351pt\hbox{$\cdot$}%
\raise  3.236pt\hbox{$\cdot$}%
\raise  3.804pt\hbox{$\cdot$}%
\raise  4.000pt\hbox{$\cdot$}%
\raise  3.804pt\hbox{$\cdot$}%
\raise  3.236pt\hbox{$\cdot$}%
\raise  2.351pt\hbox{$\cdot$}%
\raise  1.236pt\hbox{$\cdot$}%
\raise  0.000pt\hbox{$\cdot$}\-%
\raise -1.236pt\hbox{$\cdot$}%
\raise -2.351pt\hbox{$\cdot$}%
\raise -3.236pt\hbox{$\cdot$}%
\raise -3.804pt\hbox{$\cdot$}%
\raise -4.000pt\hbox{$\cdot$}%
\raise -3.804pt\hbox{$\cdot$}%
\raise -3.236pt\hbox{$\cdot$}%
\raise -2.351pt\hbox{$\cdot$}%
\raise -1.236pt\hbox{$\cdot$}%
\raise  0.000pt\hbox{$\cdot$}\-%
\raise  1.236pt\hbox{$\cdot$}%
\raise  2.351pt\hbox{$\cdot$}%
\raise  3.236pt\hbox{$\cdot$}%
\raise  3.804pt\hbox{$\cdot$}%
\raise  4.000pt\hbox{$\cdot$}%
\raise  3.804pt\hbox{$\cdot$}%
\raise  3.236pt\hbox{$\cdot$}%
\raise  2.351pt\hbox{$\cdot$}%
\raise  1.236pt\hbox{$\cdot$}%
\raise  0.000pt\hbox{$\cdot$}\-%
\raise -1.236pt\hbox{$\cdot$}%
\raise -2.351pt\hbox{$\cdot$}%
\raise -3.236pt\hbox{$\cdot$}%
\raise -3.804pt\hbox{$\cdot$}%
\raise -4.000pt\hbox{$\cdot$}%
\raise -3.804pt\hbox{$\cdot$}%
\raise -3.236pt\hbox{$\cdot$}%
\raise -2.351pt\hbox{$\cdot$}%
\raise -1.236pt\hbox{$\cdot$}!

Os argumentos de macros definidas pelo utilizador podem ser delimitados por chavetas, como
em \verb+\I\Exp{\frac{mn}{N}}+ ($\I\Exp{\frac{mn}{N}}$), ou podem terminar numa sequ\^encia
de caract\'eres definida pelo utilizador, como em \verb+\I\EXP\frac{mn}N.+
($\I\EXP\frac{mn}N.$). Ver defini\c c\~oes de \verb+\Exp+ e de \verb+\EXP+ no pre\^ambulo
desde documento; ambas t\^em um argumento, no primeiro caso delimitado por chavetas, e no
segundo \textbf{terminado} por um ponto.

Em par\'agrafos muito longos, \'e em certos casos poss\'ivel alterar o n\'umero de linhas que
eles ocupam, colocando \verb+\looseness=N+ mesmo no fim do par\'agrafo, sendo \verb.N. o
n\'umero de linhas extras que se pretendem. Por exemplo, \verb+\looseness=-1+ indica a nossa
prefer\^encia por um par\'agrafo com \textit{menos} uma linha do que o que seria normal; caso
seja poss\'\i vel, o \sigla ir\'a honrar esse nosso pedido, reduzindo a dist\^ancia entre
palavras. Tamb\'em podemos tentar aumentar o n\'umero de linhas, usando um \verb0N0 positivo.
\looseness=-1

\'E poss\'\i vel partir f\'ormulas muito grandes usando alguns pacotes da \textit{Americal
Mathematical Society} (\verb+\usepackage{amsmath}+ e \verb+\usepackage{amssymb}+, por
exemplo). Aqui vai um exemplo:
\begin{equation}
  \begin{split}
    F(z) & = \sum_{k=-\infty}^{+\infty} f(n)\,z^{-n}
             \qquad\mbox{isto~\ldots}\qquad
             \sum_{i=-\infty}^{+\infty}\sum_{j=-\infty}^{+\infty} F_{ij}
             \\
         & = \sum_{k=-\infty}^{+\infty} n^3\,z^{-n}.
  \end{split}
  \label{e:tf}
\end{equation}
Esta equa\c c\~ao tem o n\'umero~\ref{e:tf}. Note que a parte final da frase anterior foi
escrita da seguinte maneira: \verb+n\'umero~\ref{e:tf}+. O caract\'er \verb+~+ \'e
substitu\'\i do por um espa\c co e o \sigla \textbf{n\~ao pode} partir a linha nesse
s\'\i tio. Neste caso, nunca ser\'a poss\'\i vel ficar o texto ``n\'umero'' no fim de uma
linha e o texto ''\ref{e:tf}'' no in\'\i cio da linha seguinte (o que seria \textbf{muito}
deselegante). Em geral, quando uma frase termina com uma palavra (ou f�rmula matem\'atica)
pequena, \'e deselegante que essa palavra fique numa nova linha (use \verb+~+ nesses casos para
que isso n\~ao aconte\c ca).

\medskip
\'E poss\'\i vel introduzir um espa\c co vertical extra entre par\'agrafos usando as macros
\verb+\smallskip+, \verb+\medskip+ e \verb+\bigskip+. Na op\c c\~ao \texttt{final} n\~ao aparece
uma caixa preta (ver linha anterior), sempre que uma linha \'e grande de mais (sempre que isto
acontece, deve-se inserir ou eliminar texto para que deixe de acontecer).


\section{Mais alguns exemplos, agora sem qualquer explica\c c\~ao}

\begin{table}[htbp]
  \begin{center}%
    \begin{tabular}{c|c}\hline
      $n$ & $f(n)$ \\ \hline\hline
      $1$ & $1$ \\[3mm]
      $2$ & $4$ \\ \hline
    \end{tabular}$\qquad$%
    \begin{tabular}{|c|c|}\hline
      $n$ & $f(n)$ \\ \hline\hline
      $1$ & $1$ \\ \hline
      $2$ & $4$ \\ \hline
    \end{tabular}%
    \caption{Isto \'e a tabela~\ref{t:tab1}.}%
    \label{t:tab1}%
  \end{center}%
  \vspace*{5mm}%
  \tabcolsep=20mm%
  \begin{center}%
    \hspace*{-18pt}\begin{tabular}{|c|c|c|c|c|}\hline
      1 & 2 & 3 & 4 \\ \hline
    \end{tabular}\hspace*{-18pt}%
    \caption{Isto \'e a tabela~\ref{t:tab2}.}%
    \label{t:tab2}%
  \end{center}%
\end{table}

\def\LINHA#1#2{\def\ABC{#1}\def\DEF{}%
  \hbox{\strut
    \hbox to 40mm{\ifx\ABC\DEF\hss\else$\bullet$ \textbf{#1} \leaders\hbox{\,.\,}\hss\space\fi}%
    #2\relax\strut}}
\noindent\indent
\vbox{%
  \hrule height 2pt depth 0pt%
  \hbox{%
    \vrule width 2pt\kern 2mm%
    \vbox{%
      \kern 2mm%
      \LINHA{Nome:}{Z\'e Manel}%
      \LINHA{Idade:}{2}%
      \LINHA{Morada:}{Sajhd sakjhd sakdhsa kdhsa hsa sakjhd}%
      \LINHA{}{kdjsadsa kdjsakdjsa d}%
      \kern 2mm%
    }%
    \kern 2mm\vrule width 2pt%
  }%
  \hrule height 2pt depth 0pt%
}

\def\ColA{57mm}
\def\ColB{57mm}
\def\Box#1#2{\setbox0=\hbox to \ifcase#1 \ColA\else\ColB\fi{\hss #2\hss}%
  \dimen0=\ht0\advance\dimen0 by 1mm\ht0=\dimen0
  \dimen0=\dp0\advance\dimen0 by 1mm\dp0=\dimen0\box0}
\def\RBox#1#2#3{\setbox0=\hbox to \ifcase#1 \ColA\else\ColB\fi{\hss #2\hss\rlap{#3}}%
  \dimen0=\ht0\advance\dimen0 by 1mm\ht0=\dimen0
  \dimen0=\dp0\advance\dimen0 by 1mm\dp0=\dimen0\box0}
\def\XBox#1#2#3{\setbox0=\hbox to \ifcase#1 \ColA\else\ColB\fi{\hss #2\hss}%
  \dimen0=\ht0\advance\dimen0 by 1mm\ht0=\dimen0
  \dimen0=\dp0\advance\dimen0 by 1mm\dp0=\dimen0\raise#3mm\box0}
\def\SBox#1#2#3{\vbox{\Box{#1}{#2}\Box{#1}{#3}}}

\begin{table}[htb]\centering\small
  \caption{F\'ormulas relacionadas com a s\'erie cl\'assica de Fourier ($\Omega_T=\frac{2\pi}{T}$)}%
  \vspace*{2mm}
  \def\ColA{59mm}
  \def\ColB{59mm}
  \begin{tabular}{|c|c|}\hline
      \Box{0}{Dom\'\i nio dos tempos} &
      \Box{1}{Dom\'\i nio das frequ\^encias}
    \\\hline\hline
      \Box{0}{$\displaystyle f(t)=\sum_{n=-\infty}^{+\infty} f_n\,e^{i n\Omega_Tt}$} &
      \Box{1}{$\displaystyle f_n = \frac1T\int_0^T f(t)\,e^{-i n\Omega_Tt}\,dt$}
    \\\hline
      \XBox{0}{$h(t)=f(t)g(t)$}{2.5} & \SBox{1}{$h_n=f_n \ast g_n$}{$\displaystyle
        h_n=\sum_{m=-\infty}^{+\infty} f_{n-m}g_m$}
    \\\hline
      \SBox{0}{$h(t)=f(t)\ast g(t)$}{$\displaystyle h(t)=\frac1T\int_0^T
        f(\tau)g(t-\tau)\,d\tau$} &
      \XBox{1}{$h_n=f_ng_n$}{3}
    \\\hline
      \multicolumn{1}{|c}{\RBox{0}{$\displaystyle \bigl\langle f(t),g(t)\bigr\rangle =
        \frac1T\int_0^T f(t)\overline{g(t)}\,dt$}{$\mkern 2mu=$}} &
      \Box{1}{$\displaystyle \langle f_n,g_n\rangle=\sum_{n=-\infty}^{+\infty} f_n\overline{g_n}$}
    \\\hline
  \end{tabular}
\end{table}


\cleardoublepage
\chapter{``\textit{Fun}''}

Neste cap\'\i tulo limitamo-nos a apresentar uma lista dos primeiros cem
n\'u\-me\-ros primos, gerados automaticamente pelo pr\'oprio \TeX\ (exemplo, ligeiramente
modificado, extra\'\i do do livro ``The \TeX book'', escrito pelo Prof.\ Donald E. Knuth).
O c\'odigo utilizado para gerar esta lista \'e o seguinte:
\begin{verbatim}
\newif\ifprime\newif\ifunknown\newcount\n\newcount\p\newcount\d\newcount\a
\def\primes#1{2,~3\n=#1\advance\n by-2\p=5\loop\ifnum\n>0\printifprime
  \advance\p by2\repeat}
\def\printp{\ifnum\n=1\ e~\else, \fi\number\p\advance\n by-1}
\def\printifprime{\testprimality\ifprime\printp\fi}
\def\testprimality{{\d=3\global\primetrue\loop\trialdivision
  \ifunknown\advance\d by2\repeat}}
\def\trialdivision{\a=\p\divide\a by\d\ifnum\a>\d\unknowntrue\else
  \unknownfalse\fi\multiply\a by \d\ifnum\a=\p\global\primefalse
  \unknownfalse\fi}
\primes{100}.
\end{verbatim}


\section{Os primeiros cem n\'umeros primos}

\newif\ifprime\newif\ifunknown\newcount\n\newcount\p\newcount\d\newcount\a
\def\primes#1{2,~3\n=#1\advance\n by-2\p=5\loop\ifnum\n>0\printifprime\advance\p by2\repeat}%
\def\printp{\ifnum\n=1\ e~\else, \fi\number\p\advance\n by-1}%
\def\printifprime{\testprimality\ifprime\printp\fi}%
\def\testprimality{{\d=3\global\primetrue\loop\trialdivision\ifunknown\advance\d by2\repeat}}%
\def\trialdivision{\a=\p\divide\a by\d\ifnum\a>\d\unknowntrue\else\unknownfalse\fi
 \multiply\a by \d\ifnum\a=\p\global\primefalse\unknownfalse\fi}%
\primes{100}.


\cleardoublepage
\chapter{Lixo 2}

\section{Lixo 2.1}

Mais uma brincadeira com uma tabela (tabela~\ref{t:align}).

\begin{table}
  \centering
  \begin{tabular}{|c|r@{$.$}l|}
    \hline
    \Header{\textbf{ Nome do }}{\textbf{ programa }} &
    \multicolumn{2}{c|}{ \bf tempo } \\
    \hline
    abc & $10$ & $000$  \\
    def & $12$ & $0$    \\
    ghi &  $9$ & $0928$ \\
    jkl & $20$ & $0293$ \\
    \hline
  \end{tabular}
  \caption{Uma maneira poss\'\i vel de alinhar n\'umeros pela v\'\i rgula (na realidade, ponto)}
  \label{t:align}
\end{table}

\subsection{Lixo 2.1.1}

Inclus\~ao de uma figura (figura~\ref{f:seno}) gerada seguinte c\'odigo \texttt{MATLAB}:
\begin{verbatim}
  >> t=0:0.01:1;
  >> plot(t,sin(2*pi*t));
  >> title('seno');
  >> grid on
  >> print -depsc2 'example_fig.eps'
\end{verbatim}

\begin{figure}
  \centering
  \includegraphics[width=0.9\textwidth]{example_fig.eps}
  \caption{Gr\'afico de $\sin(2\pi t)$ para $0\leq t\leq 1$.}
  \label{f:seno}
\end{figure}

Idem, mas mostrando agora quadro figuras no mesmo gr\'afico. Devido ao encolhimento dos
gr\'aficos, ser\'a preciso aumentar a expessura das linhas e o tamanho da \textit{font} no
\texttt{MATLAB} (usando os comandos \verb+get+ e \verb+set+), o que n\~ao foi feito aqui para se
ver como as letras e n\'umeros ficam pequenos.
\begin{figure}
  \centering
  \begin{tabular}{|c|c|}
    \hline
      \includegraphics[width=0.4\textwidth]{example_fig.eps} a) &
      \includegraphics[width=0.4\textwidth]{example_fig.eps} b) \\
    \hline
      \includegraphics[width=0.4\textwidth]{example_fig.eps} c) &
      \includegraphics[width=0.4\textwidth]{example_fig.eps} d) \\
    \hline
  \end{tabular}
  \caption{a) descri\c c\~ao do painel do canto superior, $\ldots$}
  \label{f:seno2}
\end{figure}


\subsection{Lixo 2.1.2}

\subsubsection{Lixo 2.1.2.1}

Uma sub-sub-sec\c c\~ao!


\subsubsection{Lixo 2.1.2.2}


\section{Lixo 2.2}

Veja a tabela~\ref{t:tabela1}.

\begin{table}[hbt]
  \centering
  \begin{tabular}{c|c|c}
    $x$ & $x^2$ & $x^3$ \\\hline
    $1$ &   $1$ &   $1$ \\
    $2$ &   $4$ &   $8$ \\
    $3$ &   $9$ &  $27$ \\
    $4$ &  $16$ &  $64$ \\
    $5$ &  $25$ & $125$
  \end{tabular}
  \caption[Uma tabela!]{Uma tabela! Lixo, lixo, lixo, lixo, lixo, lixo, lixo, lixo, lixo, lixo,
    lixo, lixo, lixo, lixo, lixo, lixo, lixo, lixo, lixo, lixo, lixo, lixo, lixo, lixo, ldots}
  \label{t:tabela1}%
\end{table}

\section{Lixo 2.3}

\subsection{Lixo 2.3.1}

\subsection{Lixo 2.3.2}

\subsection{Lixo 2.3.3}

\subsection{Lixo 2.3.4}

\subsection{Lixo 2.3.5}

Veja a figura~\ref{f:figura1}.

\begin{figure}[tb]
  \centering
  \setlength{\unitlength}{1pt}
  \begin{picture}(100,100)(0,0)
    \put(-0.20,-0.20){\rule{100.40pt}{0.40pt}}
    \put(-0.20,99.80){\rule{100.40pt}{0.40pt}}
    \put(-0.20,-0.20){\rule{0.40pt}{100.40pt}}
    \put(99.80,-0.20){\rule{0.40pt}{100.40pt}}
    \put(50,50){\makebox(0,0){UMA FIGURA!}}
  \end{picture}
  \caption[Texto explicativo mais pequeno!]{Uma figura! Lixo, lixo, lixo, lixo, lixo, lixo, lixo,
    lixo, lixo, lixo, lixo, lixo, lixo, lixo, lixo, lixo, lixo, lixo, lixo, lixo, lixo, lixo,
    lixo, lixo, lixo, lixo, ldots}
  \label{f:figura1}%
\end{figure}

\subsection{Lixo 2.3.6}

\section{Lixo 2.4}

\section{Lixo 2.5}

\section{Lixo 2.6}


\cleardoublepage
\chapter{Lixo 3}

\section{Lixo 3.1}

\subsection{Lixo 3.1.1}

\subsection{Lixo 3.1.2}

\section{Lixo 3.2}

\section{Lixo 3.3}

\subsection{Lixo 3.3.1}

\subsection{Lixo 3.3.2}

\subsection{Lixo 3.3.3}

\subsection{Lixo 3.3.4}

\subsection{Lixo 3.3.5}

\section{Lixo 3.4}

\section{Lixo 3.5}

\section{Lixo 3.6}

\section{Lixo 3.7}

\section{Lixo 3.8}

\section{Lixo 3.9}


\cleardoublepage
\chapter{Conclus\~ oes}

Que conclus\~oes?

Exemplo de duas entradas da ``\textit{bib file}'':

{\footnotesize
\begin{verbatim}
@Article
{
  Eliahou-1-1993-CLBNCL,
  author = {Eliahou, Shalom},
  title = {The $3x+1$ Problem: New Lower Bounds on Nontrivial Cycle Lengths},
  journal = {Discrete Mathematics},
  year = {1993},
  volume = {118},
  number = {1--3},
  pages = {45--56}
}

@Article
{
  Garner-1981-1-OCA,
  author = {Garner, Lynn E.},
  title = {On the Collatz $3n+1$ Algorithm},
  journal = {Proceedings of the American Mathematical Society},
  year = {1981},
  volume = {82},
  number = {1},
  pages = {19--22},
  month = May
}
\end{verbatim}
}


%
% The bibliography
%
\cleardoublepage
\iffalse
  % Use this is the final version
  %  unsrt produces numbered entries, sorted by order of citation
  %  plain produces numbered entries, sorted alphabetically
  %  other styles are possible (I recommend the harvard package)
  \bibliographystyle{unsrt}
  %\bibliographystyle{plain}
  \bibliography{my-bib-file}% replace by the name of name of your .bib file
\else
  % An example (the contents of the .bbl file)
  \begin{thebibliography}{10}

  \bibitem{Eliahou-1-1993-CLBNCL}
  Shalom Eliahou.
  \newblock The $3x+1$ problem: New lower bounds on nontrivial cycle lengths.
  \newblock {\em Discrete Mathematics}, 118(1--3):45--56, 1993.

  \bibitem{Garner-1981-1-OCA}
  Lynn~E. Garner.
  \newblock On the collatz $3n+1$ algorithm.
  \newblock {\em Proceedings of the American Mathematical Society}, 82(1):19--22,
    May 1981.
  \end{thebibliography}
\fi
\cleardoublepage

\end{document}
